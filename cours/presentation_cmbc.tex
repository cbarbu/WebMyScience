% Fichier générant le PDF en latex du cours
% compilable par exemple avec 
% pdflatex presentation_cmbc.tex

\documentclass[pdftex,xcolor={table}]{beamer} % truc de base

% \documentclass[notes]{beamer} % pour avoir à la fois les transparents et les notes (alternés)
% \documentclass[notes=only]{beamer} % pour avoir seulement les notes
% \usepackage{pgfpages}
% \pgfpagesuselayout{2 on 1}[a4paper,border shrink=5mm,portrait] % pour avoir deux ``pages'' de notes sur une feuille A4
% \setbeameroption{notes on second screen}
\setbeamerfont{note page}{size=\tiny} % pour changer la taille de police des notes
\setbeamerfont{block body}{size=\small} % pour changer la taille de police des notes
% look at beamerfontthemedefault.sty for more details

\definecolor{fuchsia}{rgb}{0.5,0.5,0.9}
% \documentclass[notes=only]{beamer} % pour avoir seulement les notes (super)
% \documentclass[compress]{beamer} % rétrécie les banières du haut et du bas en étalant horizontalement les parties
% type de document 
% \documentclass[a4paper,12pt]{article} % document de type article elsevier
\usepackage[active]{srcltx}
% packages de fonts et de language
% \usepackage{textcomp} 		% besoin pour frenchb du package babel
\usepackage[english]{babel}	% le texte ajout� automatiquement dans le doc est en fran�ais sauf qu'il me fait des warnings donc zapp�
\usepackage[utf8]{inputenc} 	% autorise accents etc... n�cessite babel [fran�ais]
\usepackage[T1]{fontenc}	% permet l'utilisation du codage de caract�re "moderne"
\usepackage[normalem]{ulem}	% permet des soulignements / ratures sp�ciaux sur le texte
% \usepackage[francais]{layout}
% \usepackage[ansinew]{inputenc}	% quand je le met il me met des warning et il ne semble pas servir � grand chose, choisir entre �a et latin1, voir plus haut.

% distribution ams pour les environnements math�matiques
% \usepackage[sumlimits]{amsmath}				% package de base de AMS, n�cessaire � l'�laboration de matrices
% \usepackage{amssymb}				
% \usepackage{amsfonts}				% permettent de tracer les symboles d'ensembles
% \usepackage{amscd}
	% d'autres packages de la distribution American Mathematical Association existent et peuvent se r�v�ler n�cessaires, voir le livre LaTeX

\usepackage{natbib}				% permet qu'il fasse la bibliographie comme je veux

% packages de mise en page
\usepackage{multicol}				% d�s fois que je voudrais faire du multicolone
\usepackage{geometry}					% package de base pour la modification des marges
%\usepackage{margins}					% permet la modification locale des marges, package bidouill�
\usepackage{fancyhdr}					% gestion des en-t�te et pied de page

% packages pour les environnements sp�ciaux
\usepackage{verbatim}					% permet l'inclusion de texte non int�rpr�t� entre \begin{verbatim} et \end{verbatim}
\usepackage{graphicx}	% insertion d'images
%\usepackage{color}						% d�s fois que je voudrais autoriser la couleur
\usepackage{rotating}					% permet de faire une rotation du texte, n�cessaire pour les l�gendes verticales des figures
\usepackage{ifthen}					% pour l'utilisation d'instructions de contr�le
\usepackage{slashbox} % permet d'organiser une case d'un tableau a double entr�e avec en bas � gauche quelque chose et en haut � droite autre chose.
% syntaxe : \backslashbox{bas}{haut} -> bas\haut
% \usepackage{pstricks} %,pst-plot,pst-text,pst-tree,pst-eps,pst-fill,pst-node,pst-math} % marche pas

% d�claration de la taille du texte dans les formules
\DeclareMathSizes{10}{10}{9}{8}   % Pour un texte de taille 10 
\DeclareMathSizes{11}{11}{10}{9}   %  Pour un texte de taille 11
\DeclareMathSizes{12}{12}{11}{10}  %  Pour un texte de taille 12

%description de variables utilisables dans tout le document

% caract�ristiques du document
\author{Corentin Barbu et S�bastien Gourbi�re}

% macros
	% d�finition des "`d'apr�s"'
\newcommand{\dapres}[1]{
	{\small d'apr�s \bfseries{#1}}
	}
	% permet de num�roter les exos
\newcounter{NumExo}
\renewcommand{\theNumExo}{\Roman{NumExo}}
\newcommand{\exo}{
	\addtocounter{NumExo}{1} 
	\setcounter{NumQuestion}{0}
	{$\! \!$Exercice~\theNumExo~:\ }
	}

	% permet de num�roter et mettre en page les questions
	\newcounter{NumQuestion}
	\renewcommand{\theNumQuestion}{\arabic{NumQuestion}}
	\newcommand{\question}
		{	
			\addtocounter{NumQuestion}{1}
			{\medskip \underline{Question~\theNumQuestion~:}
			}
		}
	% raccourci pour l'environnement figure, directement ins�r� dans le texte et centr�, le premier argument est le fichier le deuxi�me est la l�gende
	\newcommand{\fig}[3]	%figure latex
	% \fig{fichier}{1}{titre} 1 correspondrait � la taille mais pas r�gl�e ici
	% appellable par \ref{fig:fichier}
		{
		\begin{figure}[h]
			\begin{center}
				\input{#1}	
				\caption{#3}
				\label{fig:#1}
			\end{center}
		\end{figure}
		}
	% idem mais pour des fichiers non latex 
	% \figx{fichier}{echelle}{l�gende}
	\newcommand{\figx}[3]	% insertion de figure, d'images non latex
		{
		\begin{figure}[h]
			\begin{center}
				\includegraphics[scale= #2 ]{#1} 	
				\caption{#3}
				\label{fig:#1}
			\end{center}
		\end{figure}
		}
	% raccourci pour la commande eqnarray*, utile notamment pour insertion des espaces de calculs pour les �l�ves
	\newcommand{\eqlin}[1]	%�quations en lignes
		{
		\begin{eqnarray*} #1 \end{eqnarray*}
		}
	% raccourci pour �crire les matrices juste avec \mat{}
	\newcommand{\mat}[1]
		{
		\begin{pmatrix} #1 \end{pmatrix}
		}% 
	% raccourci pour �crire les determinant juste avec \determ{}
	\newcommand{\determ}[1]
		{
		\begin{vmatrix} #1 \end{vmatrix}
		}
	
	\newcommand{\systeme}[1]
		{
		\left\lbrace
		        \begin{matrix} 
		        	#1 
		        \end{matrix}
		\right.
		}
	% commande equivalent (double fleche)
	\newcommand{\eqv}{\Leftrightarrow}
	% commande implique (simple fleche)
	\newcommand{\implique}{\Rightarrow}
	% commande pour dx/dt
        \newcommand{\derive}[2]{\frac{d #1}{d #2}}
	% commande pour d rond sur d rond
        \newcommand{\derivep}[2]{\frac{\partial #1}{\partial #2}}
	% pour noter la transpos�e d'une matrice
	\newcommand{\transposee}[1]{{\vphantom{#1}}^{\mathit t}{#1}}
	% pour insertion ou non des commentaires
	\newboolean{InsCom}		
	\newcommand{\ComProf}[1]
		{
		\ifthenelse{\boolean{InsCom}}
				{
				\begin{quote}
					\itshape #1
				\end{quote}
				}
				{}
		}
	% pour les notations sp�ciales de math comme les ensembles
	\newcommand{\R}{\mathbb{R}} % les r�els
	\newcommand{\Z}{\mathbb{Z}} % les entiers relatifs
	\newcommand{\N}{\mathbb{N}} % les entiers naturels
				
% options de mise en page
	% pour que �a me convienne
	%\setlength{\topmargin}{-1cm}	% permet de changer la marge d'en haut (avant m�me l'en-t�te)
	\setlength{\textheight}{22cm} % permet de g�rer la hauteur de texte et donc la marge basse

% gestion des en-t�tes et pied de page :
	% pour faire mon en t�te
	% dans le cas d'un TD
	% \TD{
\newcommand{\TD}[4]{	
	\pagestyle{fancyplain} % d�fini le style global du document comme fancy suivant ce qui suit
	% en-t�te
		\lhead{\fancyplain{}{#3}} %	gauche de l'en-t�te
		\chead{\fancyplain{\textbf{\LARGE TD n\degre #1 : #2} \\ \vspace{0.3cm}
			\textit{ #3 - #4} \vspace{0.2cm}}{TD #1} } %	centre de l'en-t�te
		\rhead{\fancyplain{}{\textit{#4}}} % Droite de l'en-t�te
		\headheight 2.5cm % modifier la taille des premiers en-t�te
	% pied de page 
		%\lfoot{pied de page gauche} %	gauche
		\cfoot{\thepage} %	centre
		%\rfoot{pied de page droit} 	%	droit
		\renewcommand{\headrulewidth}{0.4pt} % Largeur du trait de s�paration de l'en-t�te mettre 0pt pour supprimer le trait.
		\renewcommand{\plainheadrulewidth}{0.4pt} % Largeur du trait en style plain
		\renewcommand{\footrulewidth}{0pt} % Largeur du trait de s�paration du pied de page mettre 0.4 pt pour revenir � un trait "`classique"' le trait.
		\thispagestyle{plain}
		\setlength{\topmargin}{-2cm}
		} 
	% dans le cas d'un contr�le continu
\newcommand{\Titre}[3]{	
% \Titre{Titre global}{sous titre gauche}{sous-titres droit}
% mettre les lignes suivantes dans le texte de la premi�re page pour que �a passe du style de page de titre au style normal
% \setlength{\textheight}{23cm} % la hauteur de texte, � mettre avant le changement de page
% \clearpage % optionnel, si on veut que l'on passe sur une nouvelle page
% % en fait clearpage, pas si optionnel que �a si on veut que les modifs s'appliquent proprements donc le placer correctement
% \pagestyle{fancy} % permet de dire qu'on passe au style fancy tel que d�finit dans le preambule
% \setlength{\topmargin}{0cm}
% \headheight 14.5pt
	\pagestyle{fancyplain} % d�fini le style global du document comme fancy suivant ce qui suit
	% en-t�te
	% dans \fancyplain{texte dans la page de titre}{texte dans les suivantes}	
		\lhead{\fancyplain{}{#1}} %	gauche de l'en-t�te
		\chead{\fancyplain{ \textbf{\LARGE #1} \\ \vspace{0.3cm}
			\textit{ #2 - #3} \vspace{0.2cm}}{} } %	centre de l'en-t�te
		\rhead{\fancyplain{}{\textit{#3}}} % Droite de l'en-t�te
		\headheight 2.5cm % modifier la taille des premiers en-t�te
	% pied de page 
		%\lfoot{pied de page gauche} %	gauche
		\cfoot{\thepage} %	centre
		%\rfoot{pied de page droit} 	%	droit
		\renewcommand{\headrulewidth}{0.4pt} % Largeur du trait de s�paration de l'en-t�te mettre 0pt pour supprimer le trait.
		\renewcommand{\plainheadrulewidth}{0.4pt} % Largeur du trait en style plain
		\renewcommand{\footrulewidth}{0pt} % Largeur du trait de s�paration du pied de page mettre 0.4 pt pour revenir � un trait "`classique"' le trait.
                \thispagestyle{plain}
		\setlength{\topmargin}{-2cm} % marge au dessus du haut de page pour l'ensemble du document
		\setlength{\textheight}{23cm} % taille du texte pour l'ensemble du document
		}

% petit lexique de latex compl�mentaire
% \textit{texte � mettre en italique}
% \emph{texte � mettre en exergue}
% \underset{truc}{chose} met truc en-dessous de chose � utiliser avec underbrace notamment
% \underbrace{} accolade dessous
% \approx � peu pr�s �gal �
% \pm plus ou moins
% ~ espace ins�cable dans le texte


\usepackage{fancyvrb}
\usepackage{graphicx}
\usepackage{multimedia} % pour utiliser la commande movie 
\usepackage{lmodern}           % Enable Latin Modern fonts
\usepackage{dsfont} % for the double stoke in math mode
% \usepackage[hang,flushmargin]{footmisc} % indentation of footnote
% \usepackage{enumitem} % for customizing items and margins

% \usepackage{beamerthemesplit}
% \usepackage[final]{movie15}
\usepackage{times}
\usepackage{tabularx}
% \usepackage{pgf,pgfarrows,pgfnodes,pgfautomata,pgfheaps,pgfshade}
\mode<presentation>
%% style 
% keep \usecolor before \usetheme
\usecolortheme{forest} % forest is my personal green theme
% \usetheme{Warsaw}
% also consider Boadilla, Singapore or Goettingen
% also consider seahorse, seagull or native Boadilla
\usetheme[width=0.17\textwidth,hideothersubsections]{Goettingen} % customized in same directory 
\setbeamertemplate{navigation symbols}{} % pour retirer les symboles de navigation
\usefonttheme[onlylarge]{structuresmallcapsserif} % change la police des titres
\setbeamerfont{note page}{size=\tiny} % pour changer la taille de police des notes
\setbeamerfont{block body}{size=\small} % pour changer la taille de police des notes
% look at beamerfontthemedefault.sty for more details

% \usetheme{Goettingen}
\def\euro{\mbox{\raisebox{.25ex}{{\it =}}\hspace{-.5em}{\sf C}}}

% \pgfpagesuselayout{2 on 1}[a4paper,border shrink=5mm,landscape]

% pour permettre le ``voilage'' des items mentionnés
\beamertemplatetransparentcovered
% \opaqueness{90}
\setbeamercovered{still covered={\opaqueness<1->{25}},again covered={\opaqueness<1->{50}}}
% \setbeamercovered{visible} % useless ?
% \beamerboxesdeclarecolorscheme{myalert}{red}{black!5!averagebackgroundcolor}
% \beamerboxesdeclarecolorscheme{mybox}{blue}{black!5!averagebackgroundcolor}
\usepackage[absolute,showboxes,overlay]{textpos}
\textblockorigin{0cm}{0cm} % origine des positions
% \TPshowboxestrue % affiche le contour des textblock
\TPshowboxesfalse % n’affiche pas le contour des textblock

\DeclareGraphicsExtensions{.pdf}  % car pdflatex ne sait pas inclure de .eps
\graphicspath{{./}{./super\_combo/images/}} 
% \DeclareUnicodeCharacter{00B1}{±}

\title[Ma science sur le web]{
Ma science sur le web
}
\subtitle{
Développer des applications web en recherche à destination de différents publics
}
\author[Corentin Barbu]{Corentin M. Barbu}
\date{9 avril 2025}
\institute[ABIES]{INRAE et Ecole Doctorale ABIES (réseau ADUM)}

\setbeamertemplate{navigation symbols}{} % pour retirer les symboles de navigation

% \usepackage{aeguill}
\usepackage{multirow}
\usepackage{pifont}

\usepackage{bbding} % for \checkmarck
\newcommand{\Href}[2]{\href{#1}{\color{blue}#2}}
\newcommand{\ra}[0]{$\rightarrow$ }
\newcommand{\Ra}[0]{$\Rightarrow$ }
\newcommand{\cross}[0]{$\times$ } % to correspond to \checkmark
\newcommand{\ball}[0]{$\bullet$ } % to correspond to \checkmark
\newcommand{\about}[0]{$\sim $ } % to correspond to \checkmark
\renewcommand{\thefootnote}{}
\renewcommand{\footnotetext}{\tiny\color{black!10}}

\newcolumntype{M}[1]{>{\raggedright}m{#1}}
\newcolumntype{C}[1]{>{\centering}m{#1}}

\setbeamerfont{itemize/enumerate body}{size=\footnotesize}
\setbeamerfont{itemize/enumerate subbody}{size=\scriptsize}
\setbeamerfont{itemize/enumerate subsubbody}{size=\tiny}

% likes to be the last imported package

%%%%%%%%%%%%%%%%
\newenvironment{changemargin}[2]{%
\begin{list}{}{%
  \setlength{\topsep}{0pt}%
  \setlength{\leftmargin}{#1}%
  \setlength{\rightmargin}{#2}%
  \setlength{\listparindent}{\parindent}%
  \setlength{\itemindent}{\parindent}%
  \setlength{\parsep}{\parskip}%
  }%
\item[]}{\end{list}}
  %%%%Then \begin{changemargin}{-1cm}{-1cm} makes the margin 1 cm wider on either side of the page until \end{changemargin} appears.

  \newcommand{\ititre}[1]{
  {\scriptsize { #1 }\par}
  }

   \makeatletter
   \newenvironment{customlist}[2]{
   \ifnum\@itemdepth >2\relax\@toodeep\else
   \advance\@itemdepth\@ne%
   \beamer@computepref\@itemdepth%
   \usebeamerfont{itemize/enumerate \beameritemnestingprefix body}%
   \usebeamercolor[fg]{itemize/enumerate \beameritemnestingprefix body}%
   \usebeamertemplate{itemize/enumerate \beameritemnestingprefix body begin}%
   \begin{list}
     {
     \usebeamertemplate{itemize \beameritemnestingprefix item}
     }
     { \leftmargin=#1 \itemindent=#2
     \def\makelabel##1{%
     {%  
     \hss\llap{{ %
     \usebeamerfont*{itemize \beameritemnestingprefix item}%
     \usebeamercolor[fg]{itemize \beameritemnestingprefix item}##1}}%
     }%  
     }%  
     }
     \fi
     }
     {
   \end{list}
   \usebeamertemplate{itemize/enumerate \beameritemnestingprefix body end}%
   }
   \makeatother


   % Add support for \subsubsectionpage
   \def\subsubsectionname{\translate{Subsubsection}}
   \def\insertsubsubsectionnumber{\arabic{subsubsection}}
   \setbeamertemplate{subsubsection page}
   {
   \begin{centering}
     {\usebeamerfont{subsubsection name}\usebeamercolor[fg]{subsubsection name}\subsubsectionname~\insertsubsubsectionnumber}
     \vskip1em\par
     \begin{beamercolorbox}[sep=4pt,center]{part title}
       \usebeamerfont{subsubsection title}\insertsubsubsection\par
     \end{beamercolorbox}
   \end{centering}
   }
   \def\subsubsectionpage{\usebeamertemplate*{subsubsection page}}
   
   \AtBeginSection{\frame{\sectionpage}}
   \AtBeginSubsection{\frame{\subsectionpage}}
   \AtBeginSubsubsection{\frame{\subsubsectionpage}}

\usepackage{hyperref}
\hypersetup{
backref=true,     %permet d'ajouter des liens dans...
pagebackref=true, %...les bibliographies
hyperindex=true,  %ajoute des liens dans les index.
colorlinks=true, %colorise les liens
breaklinks=true,  %permet le retour à la ligne dans les liens trop longs
urlcolor= blue,   %couleur des hyperliens
% linkcolor= blue, %couleur des liens internes
bookmarks=true,   %créé des signets pour Acrobat
bookmarksopen=true,            %si les signets Acrobat sont créés,
%les afficher complètement.
pdftitle={Introduction aux tests statistiques et aux modèles linéaires}, %informations apparaissant
pdfauthor={Corentin M. Barbu},     %dans les informations du document
pdfsubject={Bases stats}          %sous Acrobat.
pdfkeywords={statistiques, métriques paysagères, élicitation, comparaison, bibliographie}, % list of keywords
}
%%%%%%%%%%%%%%%%%%%%%%%%%%%%%%%%%%%%%%%%%
\begin{document}
%%%%%%%%%%%%%%%%%%%%%%%%%%%%%%%%%%%%%%%%%
  \begin{frame}
    \titlepage
    \vfill
    ~
  \end{frame}

  \begin{frame}{Introduction}
    \begin{block}{Objectifs}
      \begin{itemize}
        \item Voir les outils de publications sur internet et comprendre leurs spécificités
      \end{itemize}
    \end{block}
    \begin{block}{Programme}
      \begin{itemize}
        \item présentations des possibilités de mise en ligne suivant la complexité du projet 
          \item application minimaliste fonctionnant en local
          \item mise en ligne de l'application minimaliste sur un serveur
        \end{itemize}
        Note: les exemples seront tous en R mais les mêmes outils sont souvent aussi disponibles sous Python
      \end{block}

  \end{frame}

  \begin{frame}{3 grands types de mises en ligne}
    \begin{block}{Site internet (plus ou moins interactif)}
      \begin{itemize}
        \item Une collection de documents fixes : Quarto (Rmarkdown) \ldots\\
          ex: \url{https://mocoriba.fr/}
        \item \ldots mais des graphiques peuvent être interactifs (plotly) \ldots\\
          ex: \url{https://plotly.com/r/getting-started/}
        \item \ldots la page peut même contenir de mini applications (shiny) \\
          ex: \url{https://quarto.org/docs/interactive/shiny/}
      \end{itemize}
    \end{block}

    \begin{block}{Une application web}
      \begin{itemize}
        \item Une interface qui communique avec un serveur \ldots \\
        \ldots pour un contenu interactif.
          ex: \href{https://ccexplorer.eu}{https://ccexplorer.eu}
      \end{itemize}
    \end{block}

    \begin{block}{Une API (Plumber)}
    \begin{itemize}
      \item Permettre la consultation de bases de données, plus ou moins traitées :
        ex: Swagger MoCoRiBA: \url{https://mocoriba.fr/mocoribaAPI/\_\_docs\_\_/}
    \end{itemize}
    \end{block}
  \end{frame}

\section{Spécificités des trois solutions}
  \begin{frame}{Quarto}
    Note: Si Rmarkdown (.Rmd) existe encore, il est progressivement remplacé par Quarto (.qmd)
    \begin{block}{Comment ça se présente}
      \begin{itemize}
        \item Quarto en ligne de commande
        \item Quarto dans R (python)
        \item aide de quarto
      \end{itemize}
    \end{block}
    \begin{block}{Avantages et inconvénients}
      \begin{itemize}
        \item qmd \ra pdf ou html
        \item Organisation générale 
        \item tables des matières automatiques
        \item Recherche dans les billets de blog et navigation blog facilitées
        \item[\cross] les html générés sont très lourds 
      \end{itemize}
    \end{block}
  \end{frame}
  \begin{frame}{Shiny}
    \begin{block}{Comment ça se présente}
      \begin{itemize}
        \item Un fichier app.R (dans le cas le plus simple)
        \item 2 fonctions : ui et server
        \item D'autres choses avant qui définissent tout ce que l'on veut en R simple
        \item possibilité d'inclure: 
          \begin{itemize}
            \item html et javascript
            \item appel à des API externes
          \end{itemize}
      \end{itemize}
    \end{block}
    \begin{block}{Avantages et inconvénients}
      \begin{itemize}
        \item Extrêmement flexible
        \item Relativement simple pour une application web (ui/server)
        \item[\cross] Assez lent
        \item[\cross] Plus complexe que R habituel, surtout sensible pour le débuggage
      \end{itemize}
    \end{block}
  \end{frame}
  \begin{frame}{Plumber}
    \begin{block}{Comment ça se présente}
      \begin{itemize}
        \item Un fichier .R chargé dans R (voir plumber.R)
        \item Génération automatique d'une documentation en ligne et interactive (Swagger)
      \end{itemize}
    \end{block}
    \begin{block}{Avantages et inconvénients}
      \begin{itemize}
        \item Extrêmement flexible
        \item Très simple à prendre en main
        \item Très rapide et relativement simple à débugger
      \end{itemize}
    \end{block}
  \end{frame}
\section{Les programmes compagnons}
  \begin{frame}{La gestion des codes}
    \begin{block}{Le protocole/programme ubiquitaire: git}
      \begin{itemize}
        \item facilite la sauvegarde
        \item permet la restauration
        \item permet le développement en parallèle de plusieurs ``branches''
      \end{itemize}
    \end{block}
  \end{frame}
  \begin{frame}{La gestion des données}
    \begin{itemize}
      \item Fichiers ouverts directement par R (rds/csv/etc.), data/
      \item Base clé-valeur : ex. REDIS 
      \item Base de données relationelle : ex: PostgresSQL
    \end{itemize}
  \end{frame}

\section{Le déploiement}
  \begin{frame}{Mise en ligne et passage à l'échelle}
    \begin{block}{Solutions de mise en ligne}
      \begin{itemize}
        \item Solutions d'hébergement d'applications (ex: SK8 \url{https://sk8.inrae.fr/} ou Shinyapps.io)
        \item Hébergement cloud/``local'' (ex: OVH)
          \begin{itemize}
            \item[\Ra] serveur en ligne (linux ?)
            \item nom de domaine libre (ovh, inrae)
            \item installation de shiny server \\
              \ra sudo apt install shiny-server
            \item placer son application au bon endroit
            \item proxy (ex:nginx)\\
              ex: redirection ccexplorer
          \end{itemize}
      \end{itemize}
    \end{block}
    \begin{block}{Passage à l'échelle}
      \begin{itemize}
        \item hébergement d'application : payer
        \item serveur géré ex: docker swarm
      \end{itemize}
    \end{block}
  \end{frame}
  \begin{frame}{Passage à l'échelle avec Docker et Traefik}
    \begin{itemize}
      \item docker
        \begin{itemize}
          \item permet la conteneurisation : ``mini machines virtuelles''
          \item un conteneur est de tailles beaucoup plus faible qu'une machine virtuelle
          \item installations dans le conteneur indépendantes du système
          \item déploiement ultra-rapide sur de nouveaux serveurs
        \end{itemize}
      \item docker swarm \\
        \begin{itemize}
          \item Permet la gestion de conteneurs multiples 
        \end{itemize}
      \item traefik 
        \begin{itemize}
          \item Permet de maintenir un utilisateur sur un même conteneur si besoin
          \item Un conteneur additionnel par lequel passent toutes les requêtes
        \end{itemize}
    \end{itemize}
  \end{frame}
  \begin{frame}{Sécurisation interne à l'application}
    \begin{itemize}
      \item ``purification'' des chaînes de caractères entrées\\
        ex: \Verb+entreePropre<-gsub(\textquotesingle[\textasciicircum{}a-zA-Z0-9]\textquotesingle,\textquotesingle\textquotesingle,entree)+
      \item Voir plus généralement les recommendations OPSWAT \url{https://www.opswat.com}
      \item authentification/gestion de droits par token
    \end{itemize}
  \end{frame}

  \begin{frame}{Sécurisation serveur}
    \begin{block}{Les classiques}
      \begin{itemize}
        \item pare-feu : ufw (linux) + généralement fourni par hébergeur cloud en amont
        \item https (letsencrypt/certbot)
        \item Antivirus (inrae: \url{https://www.withsecure.com/})
        \item fail2ban : surveillance de logs
        \item authentification (Basic Auth)
      \end{itemize}
    \end{block}
  \end{frame}

\subsection{Installation et mise à jour}
     \begin{frame}{Logiciels de synchronisation}
       \begin{itemize}
         \item Graphique : ex: FileZilla
         \item en ligne de commande linux : rsync \\
             rsync -avz repSource/* user@server:repDest/
       \end{itemize}
    \end{frame}

\section{Atelier}
  \begin{frame}{un site de démonstration}
    \url{https://github.com/cbarbu/WebMyScience}
    \includegraphics[scale=1]{webmyscience.png} \\
  \end{frame}
  
  \begin{frame}{Exercices supplémentaires possibles}
    \begin{itemize}
      \item Générer un pdf avec Quarto 
      \item Faire un blog Quarto (contents: posts dans l'en-tête)
      \item Faire une base de site Quarto (fichier \_quarto.yml)
      \item ajouter un graphique plotly dans l'application
      \item ajouter une entrée texte pour requêter l'API
      \item Ajouter une fonction dans l'API : résultat de prédiction d'un modèle basé sur le jeu de données iris
      \item Démonstration de mise en ligne sur un serveur (avec serveur OVH et Nginx)
    \end{itemize}
  \end{frame}

%%%%%%%%%%%%%%%%%%%%%%%%%%%%%%%%%%%%%%%%%
\end{document}
%%%%%%%%%%%%%%%%%%%%%%%%%%%%%%%%%%%%%%%%%
